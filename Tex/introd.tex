\chapter*{Введение}
\addcontentsline{toc}{chapter}{Введение}

Адаптивное общение --- это форма общения, которая адаптирована к чьим-либо потребностям и способностям. Она предназначена для предоставления людям возможности общаться с другими людьми, даже если они не могут участвовать в разговорной речи. Одним из способов реализации такого общения являются карточки со словами, обозначающими предмет или действие.

\textbf{Целью данной работы} является создание базы данных карточек для адаптивной коммуникации.
Для достижения поставленной цели необходимо выполнить следующие задачи:
\begin{itemize}
	\item проанализировать существующие СУБД;
    \item описать сущности проектируемой БД;
    \item выбрать необходимый инструментарий для реализации;
    \item реализовать спроектированную БД и необходимый интерфейс для работы с ней;
    \item исследовать характеристики разработанного программного обеспечения.
\end{itemize}
