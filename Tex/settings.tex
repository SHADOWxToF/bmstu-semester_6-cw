%\usepackage{cmap} % Улучшенный поиск русских слов в полученном pdf-файле
%\usepackage[T2A]{fontenc} % Поддержка русских букв
%\usepackage[utf8]{inputenc} % Кодировка utf8
%\usepackage[english,russian]{babel} % Языки: английский, русский
%\usepackage{enumitem}
%
%\usepackage[pdftex]{graphicx} % вставка рисунков
%\graphicspath{{img/}}%путь к рисункам
%
%% Дополнительное окружения для подписей
%\usepackage{array}
%\newenvironment{signstabular}[1][1]{
%	\renewcommand*{\arraystretch}{#1}
%	\tabular
%}{
%	\endtabular
%}
%
%\usepackage{threeparttable}
%
%\usepackage[14pt]{extsizes}
%
%\usepackage{caption}
%\captionsetup{labelsep=endash}
%\captionsetup[figure]{name={Рисунок}}
%
%\usepackage{amsmath}
%
%\usepackage[left=30mm,right=10mm,top=20mm,bottom=20mm]{geometry}
%
%% ГОСТ section
%\usepackage{titlesec}
%
%\titlespacing*{\chapter}{0pt}{-30pt}{8pt}
%\titleformat{\chapter}{\LARGE\bfseries}{\thechapter}{20pt}{\LARGE\bfseries}
%
%\titleformat{\section}[block]
%{\bfseries\normalsize}{\thesection}{1em}{}
%
%\titleformat{\subsection}[hang]
%{\bfseries\normalsize}{\thesubsection}{1em}{}
%\titlespacing\subsection{\parindent}{\parskip}{\parskip}
%
%\titleformat{\subsubsection}[hang]
%{\bfseries\normalsize}{\thesubsubsection}{1em}{}
%\titlespacing\subsubsection{\parindent}{\parskip}{\parskip}
%
%
%%\titleformat{\section}{\Large\bfseries}{\thesection}{20pt}{\Large\bfseries}
%
%%
%% \usepackage{titlesec}
%% \titleformat{\section}
%% {\normalsize\bfseries}
%% {\thesection}
%% {1em}{}
%% \titlespacing*{\chapter}{0pt}{-30pt}{8pt}
%% \titlespacing*{\section}{\parindent}{*4}{*4}
%% \titlespacing*{\subsection}{\parindent}{*4}{*4}
%
%% ГОСТ изображения и таблицы
%\usepackage{caption}
%\captionsetup[figure]{name={Рисунок},labelsep=endash}
%\captionsetup[table]{singlelinecheck=false, labelsep=endash}
%%
%
%
%\usepackage{setspace}
%\onehalfspacing % Полуторный интервал
%
%\frenchspacing
%\usepackage{indentfirst} % Красная строка
%
%
%
%\usepackage{listings}
%
%
%\lstset{
%	basicstyle=\footnotesize\ttfamily,
%%	language=[Sharp]C, % Или другой ваш язык -- см. документацию пакета
%%	commentstyle=\color{comment},
%	numbers=left,
%%	numberstyle=\tiny\color{plain},
%	numbersep=5pt,
%	tabsize=4,
%	extendedchars=\true,
%	breaklines=true,
%%	keywordstyle=\color{blue},
%	frame=single,
%%	stringstyle=\ttfamily\color{string}\ttfamily,
%	showspaces=false,
%	showtabs=false,
%	xleftmargin=17pt,
%	framexleftmargin=17pt,
%	framexrightmargin=5pt,
%	framexbottommargin=4pt,
%	showstringspaces=false,
%	inputencoding=utf8x,
%	keepspaces=true
%}
%
%%\lstset
%%{
%%%	language=C,
%%	basicstyle=\small\sffamily,
%%	numbers=left,
%%	frame=single,
%%	tabsize=4,
%%	breaklines=true
%%}
%
%\usepackage{pgfplots}
%\usetikzlibrary{datavisualization}
%\usetikzlibrary{datavisualization.formats.functions}
%
%\usepackage{dcolumn}
%
%
%\usepackage{graphicx}
%
%\usepackage[justification=centering]{caption} % Настройка подписей float объектов
%
%\usepackage[unicode,pdftex]{hyperref} % Ссылки в pdf
%\hypersetup{hidelinks}
%
%\usepackage{csvsimple}
%
%\usepackage{ragged2e}
%
%\newcommand{\code}[1]{\texttt{#1}}
%
%\usepackage{array, multirow, bigdelim, makecell} 
%
%\usepackage{tikz}
%\usetikzlibrary{graphs}
%\usepackage[normalem]{ulem}
%\usepackage{anyfontsize}
%
%\usepackage{pdfpages} % объединение с файлами формата pdf
%\usepackage[figure,table]{totalcount} % Подсчет изображений, таблиц
%\usepackage{lastpage} % Для подсчета числа страниц

\usepackage{cmap} % Улучшенный поиск русских слов в полученном pdf-файле
\usepackage[T2A]{fontenc} % Поддержка русских букв
\usepackage[utf8]{inputenc} % Кодировка utf8
\usepackage[english,russian]{babel} % Языки: английский, русский
\usepackage{enumitem}

\usepackage[pdftex]{graphicx} % вставка рисунков
\graphicspath{{img/}}%путь к рисункам

% Дополнительное окружения для подписей
\usepackage{array}
\newenvironment{signstabular}[1][1]{
	\renewcommand*{\arraystretch}{#1}
	\tabular
}{
	\endtabular
}

\usepackage{threeparttable}

\usepackage[14pt]{extsizes}

\usepackage{caption}
\captionsetup{labelsep=endash}
\captionsetup[figure]{name={Рисунок}}

\usepackage{amsmath}

\usepackage[left=30mm,right=10mm,top=20mm,bottom=20mm]{geometry}

% ГОСТ section
\usepackage{titlesec}

\titleformat{\section}[block]
{\bfseries\normalsize\filcenter}{\thesection}{1em}{}

\titleformat{\subsection}[hang]
{\bfseries\normalsize}{\thesubsection}{1em}{}
\titlespacing\subsection{\parindent}{\parskip}{\parskip}

\titleformat{\subsubsection}[hang]
{\bfseries\normalsize}{\thesubsubsection}{1em}{}
\titlespacing\subsubsection{\parindent}{\parskip}{\parskip}


% ГОСТ изображения и таблицы
\usepackage{caption}
\captionsetup[figure]{name={Рисунок},labelsep=endash}
\captionsetup[table]{singlelinecheck=false, labelsep=endash}
%


\usepackage{setspace}
\onehalfspacing % Полуторный интервал

\frenchspacing
\usepackage{indentfirst} % Красная строка

\titlespacing*{\chapter}{0pt}{-30pt}{8pt}
\titleformat{\chapter}{\LARGE\bfseries}{\thechapter}{20pt}{\LARGE\bfseries}
\titleformat{\section}{\Large\bfseries}{\thesection}{20pt}{\Large\bfseries}

\usepackage{listings}


\lstset{
		basicstyle=\footnotesize\ttfamily,
	%	language=[Sharp]C, % Или другой ваш язык -- см. документацию пакета
	%	commentstyle=\color{comment},
		numbers=left,
	%	numberstyle=\tiny\color{plain},
		numbersep=5pt,
		tabsize=4,
		extendedchars=\true,
		breaklines=true,
	%	keywordstyle=\color{blue},
		frame=single,
	%	stringstyle=\ttfamily\color{string}\ttfamily,
		showspaces=false,
		showtabs=false,
		xleftmargin=17pt,
		framexleftmargin=17pt,
		framexrightmargin=5pt,
		framexbottommargin=4pt,
		showstringspaces=false,
		inputencoding=utf8x,
		keepspaces=true
	}
%\lstset
%{
%	language=C,
%	basicstyle=\small\sffamily,
%	numbers=left,
%	frame=single,
%	tabsize=4,
%	breaklines=true
%}

\usepackage{pgfplots}
\usetikzlibrary{datavisualization}
\usetikzlibrary{datavisualization.formats.functions}

\usepackage{dcolumn}


\usepackage{graphicx}

\usepackage[justification=centering]{caption} % Настройка подписей float объектов

\usepackage[unicode,pdftex]{hyperref} % Ссылки в pdf
\hypersetup{hidelinks}

\usepackage{csvsimple}

\usepackage{ragged2e}

\newcommand{\code}[1]{\texttt{#1}}

\usepackage{array, multirow, bigdelim, makecell} 

\usepackage{tikz}
\usetikzlibrary{graphs}
\usepackage[normalem]{ulem}
\usepackage{anyfontsize}

\usepackage{pdfpages} % объединение с файлами формата pdf
\usepackage[figure,table]{totalcount} % Подсчет изображений, таблиц
\usepackage{lastpage} % Для подсчета числа страниц

\usepackage{amsfonts} % красивые буквы для обозначения множеств чисел